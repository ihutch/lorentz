\documentclass[12pt]{article}
\begin{document}
\title{Use of Matlab Particle Orbit Demo Code}
\date{Course 22.611j Plasma Learning Aid}
\author{Ian Hutchinson}
\maketitle

\section{What is it?}

A simple matlab code which calculates particle orbits in a selection
of specified magnetic and electric fields. Written by Swedish plasma
physicists as a teaching aid. They have graciously made it publically
available. 

Nothing concerning this code is required for the course.  I personally
find it less useful conceptually than would be optimum, but you might
enjoy playing around with it.

\section{How to?}
Type in to the terminal the stuff in \verb!fixed width font! below:
\begin{enumerate}
\item Log in to athena
\item \verb!add 22.611j!
\item \verb!cd /mit/22.611j/particle_orbits/jaun!
\item \verb!add matlab!
\item \verb!matlab!
\item In the matlab window that will pop up: \verb!spm!
\item Tinker with the buttons and watch the plots.
\end{enumerate}

\noindent
 Further instructions are in the html
file \verb!matlab_spm.html! in the directory.
\end{document}